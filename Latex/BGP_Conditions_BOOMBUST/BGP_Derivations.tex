\documentclass{article}
\usepackage[utf8]{inputenc}
\usepackage[english]{babel}
\usepackage[a4paper,top=2.5cm,bottom=2.5cm,left=2.5cm,right=2.5cm,%
bindingoffset=0mm]{geometry}
\usepackage{amssymb}
\usepackage{amsmath}
\newtheorem{prop}{Proposition}
\newtheorem{lemma}{Lemma}
\newenvironment{proof}[1][Proof]{\begin{trivlist}
\item[\hskip \labelsep {\bfseries #1}]}{\end{trivlist}}
\newcommand{\qed}{\nobreak \ifvmode \relax \else
      \ifdim\lastskip<1.5em \hskip-\lastskip
      \hskip1.5em plus0em minus0.5em \fi \nobreak
      \vrule height0.75em width0.75em depth0em\fi}
\usepackage{tikz}
\usepackage{graphicx}
\usepackage{rotating}
\usepackage{float}
\linespread{1.3}
\raggedbottom




%
\font\reali=msbm10 at 12pt
% subsets of real numbers
\newcommand{\real}{\hbox{\reali R}}
\newcommand{\realp}{\hbox{\reali R}_{\scriptscriptstyle +}}
\newcommand{\realpp}{\hbox{\reali R}_{\scriptscriptstyle ++}}
\newcommand{\R}{\mathbb{R}}
\DeclareMathOperator{\E}{\mathbb{E}}
%

\title{BGP Model\\Some Derivations}
\author{Marco Brianti\\Vito Cormun}
\date{Spring 2019}

\begin{document}

\maketitle

\section{Model}\label{sec:model}

Consider the following model with two states variables. The first equation describes the law of motion of investment $I_t$,
\begin{equation}\label{eq:LOMinvestment}
I_t = \alpha_{1,1} I_{t-1} + \alpha_{1,2} K_{t} + s_t^I
\end{equation}
while the second equation describes the law of motion of capital $K_t$
\begin{equation}\label{eq:LOMcapital}
K_t =  I_{t-1} + \alpha_{2,2} K_{t-1} + s_t^K.
\end{equation}
Notice that $s_t^I$ and $s_t^K$ are two disturbances such that $s_t^I \perp s_{\tau}^K$ for all $t$ and $\tau$. In addition, we also assume that $s_t^j \perp s_{\tau}^j$ for all $t$ and $\tau$ for $j \in \{ I,K\}$.\footnote{It can be easily derived that in steady state, both investment and capital are equal to $I_{ss} = K_{ss} = 0$. Without loss of generality, this result simplifies the analysis because levels outside of steady state are also deviations from steady state.}

The system is linear and it can be written in a more compactly as
\begin{equation}\label{eq:matrix_system_expanded}
\begin{pmatrix}
I_t \\
K_t
\end{pmatrix} = \begin{pmatrix}
\alpha_{1,1} & \alpha_{1,2} \\
\alpha_{2,1} & \alpha_{2,2}
\end{pmatrix}\begin{pmatrix}
I_{t-1} \\
K_{t-1}
\end{pmatrix} + \begin{pmatrix}
s_t^I \\ 
s_t^K
\end{pmatrix}
\end{equation}
which is
\begin{equation}\label{eq:matrix_system_compact}
X_t = A X_{t-1} + s_t
\end{equation}
For the rest of the document, we will assume that both $\alpha_{1,1}$ and $\alpha_{2,2}$ are positive.\footnote{The economic interpretation of this assumption is straightforward. A positive value of $\alpha_{2,2}$ means that a part of capital in the previous period is not fully depreciated and survived as an endowment in the subsequent period. Instead, $\alpha_{1,1}$ positive suggests that Equation \ref{eq:LOMinvestment} is the reduced form of a setting with dynamic strategic complementary of investment. If investment has been positive in the past then it is convenient to invest more also today.}

\section{Stability Conditions}

In order to study the stability conditions, we parametrically evaluate the eigenvalues $\lambda_1$ and $\lambda_2$ of matrix $A$. In other words, we need to solve the following problem,
$$
\det(A - \lambda I) = 0
$$
which can be rewritten as,
$$
\det \begin{pmatrix}
\alpha_{1,1} - \lambda & \alpha_{1,2} \\
\alpha_{2,1} & \alpha_{2,2} - \lambda
\end{pmatrix} = 0
$$
which is, 
\begin{equation*}
\begin{aligned}
  0 &= (\alpha_{1,1} - \lambda)(\alpha_{2,2} - \lambda) - \alpha_{1,2}\alpha_{2,1} \\
  &= \alpha_{1,1} \alpha_{2,2} + \lambda^2 - (\alpha_{1,1} + \alpha_{2,2}) \lambda - \alpha_{1,2}\alpha_{2,1} \\
   &= \lambda^2  - (\alpha_{1,1} + \alpha_{2,2}) \lambda + \alpha_{1,1} \alpha_{2,2} - \alpha_{1,2}\alpha_{2,1}
\end{aligned}
\end{equation*}
Solving over $\lambda$ yields,
\begin{equation}\label{eq:lambda_sol}
\begin{aligned}
\lambda_{1,2} &= \frac{1}{2} \bigg[ (\alpha_{1,1} + \alpha_{2,2})  \pm \sqrt{  (\alpha_{1,1} + \alpha_{2,2})^2 - 4(\alpha_{1,1} \alpha_{2,2} - \alpha_{1,2}\alpha_{2,1}) }  \bigg] \\
&= \frac{1}{2} \bigg[ (\alpha_{1,1} + \alpha_{2,2})  \pm \sqrt{  (\alpha_{1,1} - \alpha_{2,2})^2 + 4 \alpha_{1,2}\alpha_{2,1} }  \bigg] \\
\end{aligned}
\end{equation}

\begin{prop}\label{prop:complex_eigen}
Sufficient condition to have complex eigenvalues is to have $\alpha_{1,2}$ and $\alpha_{2,1}$ with a different sign. 
\end{prop}

\begin{proof}
Proof follows from Equation \ref{eq:lambda_sol}. In particular, $(\alpha_{1,1} - \alpha_{2,2})^2 + 4 \alpha_{1,2}\alpha_{2,1}$ might be negative if and only if the sign of $\alpha_{1,2}$ is different than the sign of $\alpha_{2,1}$.
\end{proof}

\begin{prop}\label{prop:real_eigen}
Sufficient condition to have real eigenvalues is to have $\alpha_{1,2}$ and $\alpha_{2,1}$ with the same sign.
\end{prop}

\begin{proof}
	Proof straightforwardly follows from Proposition \ref{prop:complex_eigen}.
\end{proof}

\begin{prop}\label{prop:stability_one}
	If $\alpha_{1,2}\alpha_{2,1} = 0$, then the system is stable (both $|\lambda_{1,2}| < 1$) if and only if both $\alpha_{1,1}$ and $\alpha_{2,2}$ are smaller than one.\footnote{Notice that we assumed both $\alpha_{1,1}$ and $\alpha_{2,2}$ to be non negative in Section \ref{sec:model}.}
\end{prop}

\begin{proof}
Since $\lambda_{1,2} = \frac{1}{2} [ \alpha_{1,1} + \alpha_{2,2}  \pm ( \alpha_{1,1} - \alpha_{2,2} ) ]$ then we have that $\lambda_1 = \alpha_{1,1}$ and $\lambda_2 = \alpha_{2,2}$. If both $\alpha_{1,1}$ and $\alpha_{2,2}$ are smaller than one then the system is automatically stable. On the other hand, if the system is stable - $|\lambda_{1,2}| < 1$ - then it must be the case that both $\alpha_{1,1}$ and $\alpha_{2,2}$ are smaller than one.
\end{proof}

\section{Shock Dependent Cyclical responses}

From now on we will assume that $\alpha_{1,2} \alpha_{2,1} < 0$, $\lambda_{1,2}$ are real and their module is smaller than one. In particular, I will assume that $\alpha_{1,2} < 0$ and $\alpha_{2,1} > 0$.

\subsection{Investment-Specific Shock}

Assume in period $t$, when the system is in steady state, that $s_t^I = 1$. Impact responses at $t$ of both variables are,
\begin{equation}\label{eq:I_responses_zero}
\begin{pmatrix}
I_t \\
K_t
\end{pmatrix} = \begin{pmatrix}
1 \\
0
\end{pmatrix} = \begin{pmatrix}
\alpha_{1,1} & \alpha_{1,2} \\
\alpha_{2,1} & \alpha_{2,2}
\end{pmatrix}\begin{pmatrix}
0 \\
0
\end{pmatrix} + \begin{pmatrix}
1 \\ 
0
\end{pmatrix}
\end{equation} 
At time $t + 1$, dynamic responses are,
\begin{equation}\label{eq:I_responses_one}
\begin{pmatrix}
I_{t+1} \\
K_{t+1}
\end{pmatrix} = \begin{pmatrix}
\alpha_{1,1} \\
\alpha_{2,1}
\end{pmatrix} = \begin{pmatrix}
\alpha_{1,1} & \alpha_{1,2} \\
\alpha_{2,1} & \alpha_{2,2}
\end{pmatrix}\begin{pmatrix}
1 \\
0
\end{pmatrix} + \begin{pmatrix}
0 \\ 
0
\end{pmatrix}
\end{equation} 
At time $t + 2$, dynamic responses are,
\begin{equation}\label{eq:I_responses_two}
\begin{pmatrix}
I_{t+2} \\
K_{t+2}
\end{pmatrix} = \begin{pmatrix}
\alpha_{1,1}^2 + \alpha_{1,2}\alpha_{2,1} \\
\alpha_{2,1}\alpha_{1,1} + \alpha_{2,2}\alpha_{2,1}
\end{pmatrix} = \begin{pmatrix}
\alpha_{1,1} & \alpha_{1,2} \\
\alpha_{2,1} & \alpha_{2,2}
\end{pmatrix}\begin{pmatrix}
\alpha_{1,1} \\
\alpha_{2,1}
\end{pmatrix} 
\end{equation} 
At time $t + 3$, dynamic responses are,
\begin{equation}\label{eq:I_responses_three}
\begin{aligned}
\begin{pmatrix}
I_{t+3} \\
K_{t+3}
\end{pmatrix} &= \begin{pmatrix}
\alpha_{1,1}(\alpha_{1,1}^2 + \alpha_{1,2}\alpha_{2,1}) + \alpha_{1,2}(\alpha_{2,1}\alpha_{1,1} + \alpha_{2,2}\alpha_{2,1})  \\
\alpha_{2,1}(\alpha_{1,1}^2 + \alpha_{1,2}\alpha_{2,1}) + \alpha_{2,2}(\alpha_{2,1}\alpha_{1,1} + \alpha_{2,2}\alpha_{2,1})
\end{pmatrix} \\
&= \begin{pmatrix}
\alpha_{1,1} & \alpha_{1,2} \\
\alpha_{2,1} & \alpha_{2,2}
\end{pmatrix}\begin{pmatrix}
\alpha_{1,1}^2 + \alpha_{1,2}\alpha_{2,1} \\
\alpha_{2,1}\alpha_{1,1} + \alpha_{2,2}\alpha_{2,1}
\end{pmatrix} 
\end{aligned}
\end{equation} 

\subsection{Capital-Specific Shock}

Assume in period $t$, when the system is in steady state, that $s_t^K = 1$. Impact responses at $t$ of both variables are,
\begin{equation}\label{eq:K_responses_zero}
\begin{pmatrix}
I_t \\
K_t
\end{pmatrix} = \begin{pmatrix}
0 \\
1
\end{pmatrix} = \begin{pmatrix}
\alpha_{1,1} & \alpha_{1,2} \\
\alpha_{2,1} & \alpha_{2,2}
\end{pmatrix}\begin{pmatrix}
0 \\
0
\end{pmatrix} + \begin{pmatrix}
0 \\ 
1
\end{pmatrix}
\end{equation} 
At time $t + 1$, dynamic responses are,
\begin{equation}\label{eq:K_responses_one}
\begin{pmatrix}
I_{t+1} \\
K_{t+1}
\end{pmatrix} = \begin{pmatrix}
\alpha_{1,2} \\
\alpha_{2,2}
\end{pmatrix} = \begin{pmatrix}
\alpha_{1,1} & \alpha_{1,2} \\
\alpha_{2,1} & \alpha_{2,2}
\end{pmatrix}\begin{pmatrix}
0 \\
1
\end{pmatrix} + \begin{pmatrix}
0 \\ 
0
\end{pmatrix}
\end{equation} 
At time $t + 2$, dynamic responses are,
\begin{equation}\label{eq:K_responses_two}
\begin{pmatrix}
I_{t+2} \\
K_{t+2}
\end{pmatrix} = \begin{pmatrix}
\alpha_{1,1}\alpha_{1,2} + \alpha_{1,2}\alpha_{2,2} \\
\alpha_{1,2}\alpha_{2,1} + \alpha_{2,2}^2
\end{pmatrix} = \begin{pmatrix}
\alpha_{1,1} & \alpha_{1,2} \\
\alpha_{2,1} & \alpha_{2,2}
\end{pmatrix}\begin{pmatrix}
\alpha_{1,2} \\
\alpha_{2,2}
\end{pmatrix} 
\end{equation} 
At time $t + 3$, dynamic responses are,
\begin{equation}\label{eq:K_responses_three}
\begin{aligned}
\begin{pmatrix}
I_{t+3} \\
K_{t+3}
\end{pmatrix} &= \begin{pmatrix}
\alpha_{1,1}(\alpha_{1,1}\alpha_{1,2} + \alpha_{1,2}\alpha_{2,2}) + \alpha_{1,2}(\alpha_{1,2}\alpha_{2,1} + \alpha_{2,2}^2)\\
\alpha_{2,1}(\alpha_{1,1}\alpha_{1,2} + \alpha_{1,2}\alpha_{2,2}) + \alpha_{2,2}(\alpha_{1,2}\alpha_{2,1} + \alpha_{2,2}^2)
\end{pmatrix} \\
&= \begin{pmatrix}
\alpha_{1,1} & \alpha_{1,2} \\
\alpha_{2,1} & \alpha_{2,2}
\end{pmatrix}\begin{pmatrix}
\alpha_{1,1}\alpha_{1,2} + \alpha_{1,2}\alpha_{2,2} \\
\alpha_{1,2}\alpha_{2,1} + \alpha_{2,2}^2
\end{pmatrix}
\end{aligned}
\end{equation} 
At time $t + 4$, dynamic responses are,
\begin{equation}\label{eq:K_responses_four}
\begin{aligned}
&\begin{pmatrix}
I_{t+4} \\
K_{t+4}
\end{pmatrix} =\\ &\begin{pmatrix}
\alpha_{1,1}[\alpha_{1,1}(\alpha_{1,1}\alpha_{1,2} + \alpha_{1,2}\alpha_{2,2}) + \alpha_{1,2}(\alpha_{1,2}\alpha_{2,1} + \alpha_{2,2}^2)] + \alpha_{1,2}[\alpha_{2,1}(\alpha_{1,1}\alpha_{1,2} + \alpha_{1,2}\alpha_{2,2}) + \alpha_{2,2}(\alpha_{1,2}\alpha_{2,1} + \alpha_{2,2}^2)]\\
\alpha_{2,1}[\alpha_{1,1}(\alpha_{1,1}\alpha_{1,2} + \alpha_{1,2}\alpha_{2,2}) + \alpha_{1,2}(\alpha_{1,2}\alpha_{2,1} + \alpha_{2,2}^2)] + \alpha_{2,2}[\alpha_{2,1}(\alpha_{1,1}\alpha_{1,2} + \alpha_{1,2}\alpha_{2,2}) + \alpha_{2,2}(\alpha_{1,2}\alpha_{2,1} + \alpha_{2,2}^2)]
\end{pmatrix} \\
&= \begin{pmatrix}
\alpha_{1,1} & \alpha_{1,2} \\
\alpha_{2,1} & \alpha_{2,2}
\end{pmatrix}\begin{pmatrix}
\alpha_{1,1}(\alpha_{1,1}\alpha_{1,2} + \alpha_{1,2}\alpha_{2,2}) + \alpha_{1,2}(\alpha_{1,2}\alpha_{2,1} + \alpha_{2,2}^2)\\
\alpha_{2,1}(\alpha_{1,1}\alpha_{1,2} + \alpha_{1,2}\alpha_{2,2}) + \alpha_{2,2}(\alpha_{1,2}\alpha_{2,1} + \alpha_{2,2}^2)
\end{pmatrix}
\end{aligned}
\end{equation} 

\begin{prop}
Investment $I_t$ displays a shock-dependent cyclical dynamics if
\begin{enumerate}
	\item $\alpha_{1,1}^2 < |\alpha_{1,2} \alpha_{2,1}|$
	\item $\alpha_{2,2}^2 > |\alpha_{1,2} \alpha_{2,1}|$
\end{enumerate}
\end{prop}

\begin{proof}
Notice that after an investment-specific shock $s_t^I$, responses of $I_t$ are,
\begin{equation*}
\begin{aligned}
I_t     &= 1   \\
I_{t+1} &= \alpha_{1,1}   \\ 
I_{t+2} &=  \alpha_{1,1}^2 + \alpha_{1,1} \alpha_{2,1}  \\ 
I_{t+3} &=  \alpha_{1,1}(\alpha_{1,1}^2 + \alpha_{1,2}\alpha_{2,1}) + \alpha_{1,2}(\alpha_{2,1}\alpha_{1,1} + \alpha_{2,2}\alpha_{2,1})  \\ 
\end{aligned}
\end{equation*}	
which implies that both $I_t$ and $I_{t+1}$ are positive by construction. However, $I_{t+2}$ and $I_{t+3}$ are negative. In particular, $I_{t+2}$ is negative because $\alpha_{1,1}^2 < |\alpha_{1,2} \alpha_{2,1}|$. Moreover, $I_{t+3}$ is negative because both $\alpha_{1,1}(\alpha_{1,1}^2 + \alpha_{1,2}\alpha_{2,1})$ and $\alpha_{1,2}(\alpha_{2,1}\alpha_{1,1} + \alpha_{2,2}\alpha_{2,1})$ are negative. This is already sufficient to show that investment $I_t$ displays a cyclical dynamics after an investment-specific shock, $s_t^I$.

Conversely, after a capital-specific shock $s_t^K$, responses of $I_t$ are,
\begin{equation*}
\begin{aligned}
I_t     &= 0   \\
I_{t+1} &= \alpha_{1,2}   \\ 
I_{t+2} &=  \alpha_{1,1} \alpha_{1,2} + \alpha_{1,2} \alpha_{2,2}  \\ 
I_{t+3} &=  \alpha_{1,1}(\alpha_{1,1}^2 + \alpha_{1,2}\alpha_{2,2}) + \alpha_{1,2}(\alpha_{1,2}\alpha_{2,1} + \alpha_{2,2}^2)  \\ 
\end{aligned}
\end{equation*}	
\end{proof}




\end{document}

